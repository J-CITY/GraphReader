\chapter*{Введение}
\addcontentsline{toc}{chapter}{Введение}
\section{Глоссарий}
\addcontentsline{toc}{section}{Глоссарий}
\begin{itemize}
  \item \textbf{OSM} (от англ. <<Open Street Map>> -- <<Открытая карта улиц>>) --- некоммерческий веб-картографический проект по созданию подробной свободной и бесплатной географической карты мира.
  \item \textbf{TinyXML} --- парсер файлов формата <<XML>> для языка C++ с открытым исходным кодом.
  \item \textbf{УДС} --- улично-дорожная сеть.
\end{itemize}

\section{Описание предметной области}
\addcontentsline{toc}{section}{Описание предметной области}
\noindent\indent Более десяти лет развивается проект Open Street Map,
ведущий грандиозную открытую базу данных картографической информации в масштабе
всей планеты Земля. С точки зрения конечного пользователя, ориентированного на
транспортное моделирование, основным продуктом проекта OSM является экспорт из
этой базы данных в формате OSM XML, содержащий значительный объем информации об
УДС различных регионов. Не менее важно, что эти данные содержат также и информацию
об объектах, являющихся точками притяжения и/или генерации трафика (кинотеатры,
музеи, магазины, стадионы, вокзалы и пр.), что является существенной компонентой
моделей транспортных систем.\par
  OSM XML снимок базы данных проекта является полной картой всей пла-
неты (Planet.osm) и в несжатом виде занимает объем в несколько сотен гига-
байт. Обновляемые еженедельно и сгруппированные по странам и регионам кар-
ты в форматах OSM XML можно загрузить, например, на сайте проекта GeoFabrik.
На портале GIS-Lab.info доступны регулярно обновляемые наборы карт
регионов Российской Федерации.\par
  В настоящей работе мы попытались написать программу для упрощения
работы с OSM-данными путем уменьшения объема исходных данных и рассчету некоторой
дополнительной информации (такой, как расстояние между узлами).
\section{Неформальная постановка задачи}
\addcontentsline{toc}{section}{Неформальная постановка задачи}
\noindent\indent Задачей данной работы является освоение формата данных проекта
Open Street Map (OSM) и разработка портабельной программы построения классической
графовой модели $G = (V, E, W)$ по OSM-данным.
